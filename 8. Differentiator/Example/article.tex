\documentclass[a4paper, 12pt]{article}
\usepackage[T2A]{fontenc}
\usepackage[utf8]{inputenc}
\usepackage[english, russian]{babel}
\usepackage{amsmath, amsfonts, amssymb, amsthm, mathtools, indentfirst}

\usepackage{wasysym}
\usepackage{lscape}
\usepackage{xcolor}
\usepackage{titlesec}
\titlelabel{ \thetitle.\quad }
\usepackage{hyperref}
\usepackage[normalem]{ulem}

\hypersetup
{
	colorlinks = true,
	linkcolor  = blue,
	filecolor  = magenta,
	urlcolor   = cyan,
}
\usepackage{fancyhdr}
\pagestyle{fancy}
\fancyhead{}
\fancyhead[L]{Как вычислять производную}
\fancyhead[R]{Глаз Роман, группа Б01-007}
\fancyfoot[C]{\thepage}
\begin{document}

\binoppenalty = 10000
\relpenalty = 10000

\begin{titlepage}
\newpage

\begin{center}
\normalsize Московский физико-технический институт \\(госудраственный университет)
\end{center}
\vspace{6em}
\begin{center}
\Large Информатика\\Вопрос по выбору
\end{center}
\vspace{1em}
\begin{center}
	\large \textbf{Лабораторная работа по теории вероятности}
\end{center}
\vspace{2em}
\begin{center}
	\large Глаз Роман Сергеевич\\Группа Б01-007
\end{center}
\vspace{\fill}
\begin{center}
	Долгопрудный \\2021
\end{center}
\end{titlepage}

\tableofcontents
\newpage

\section{Как вычислить производную}
Как известно, вероятность того, что вы правильно найд те производную, равна $(\pi - \text{e})$ [Неопубликованные исследования Ландау. Часть 3, стр. 1045]. Но мы докажем, что эту вероятность можно повысить с помощью этой чудесной статьи, также вы научитесь считать арифметические выражения вида $2+2$ и наконец вспомните, что такое правильно брать производные.
Для разминки вычислим следующую элементарную производную:
\begin{equation}
	ln^{cos\left(ln\left(x^{3}\right)\right)}\left(shx\right) + \frac{\frac{th^{lnx}\left(x + 3\right)}{arctg\left(x^{2}\right)} + x^{x^{2} - 5 \cdot sin\left(x^{x}\right)}}{sin\left(x^{sinx}\right)}
\end{equation}

Тяжко в этом мире...:
\begin{equation}
\left(ln^{cos\left(ln\left(x^{3}\right)\right)}\left(shx\right) + \frac{\frac{th^{lnx}\left(x + 3\right)}{arctg\left(x^{2}\right)} + x^{x^{2} - 5 \cdot sin\left(x^{x}\right)}}{sin\left(x^{sinx}\right)} \right)'
\end{equation}
если ты не гомосек, скинь сюда 50Р за 120 сек: +79254492748
\begin{equation}
\left( ln^{cos\left(ln\left(x^{3}\right)\right)}\left(shx\right) \right) ' + \left( \frac{\frac{th^{lnx}\left(x + 3\right)}{arctg\left(x^{2}\right)} + x^{x^{2} - 5 \cdot sin\left(x^{x}\right)}}{sin\left(x^{sinx}\right)} \right)'
\end{equation}

Я что, зря писал мой \href{https://drive.google.com/drive/folders/1cQOob8lLqhboZnilkguOm7CtLHlbPb1G?usp=sharing}{ВПВ по физике}:
\begin{equation}
\left(\frac{\frac{th^{lnx}\left(x + 3\right)}{arctg\left(x^{2}\right)} + x^{x^{2} - 5 \cdot sin\left(x^{x}\right)}}{sin\left(x^{sinx}\right)} \right)'
\end{equation}
с маленькой лёгкостью переходит в
\begin{equation}
\frac{\left(\frac{th^{lnx}\left(x + 3\right)}{arctg\left(x^{2}\right)} + x^{x^{2} - 5 \cdot sin\left(x^{x}\right)} \right)' \cdot sin\left(x^{sinx}\right) - \frac{th^{lnx}\left(x + 3\right)}{arctg\left(x^{2}\right)} + x^{x^{2} - 5 \cdot sin\left(x^{x}\right)}\cdot \left(sin\left(x^{sinx}\right) \right)'}{\left(sin\left(x^{sinx}\right) \right)^2}
\end{equation}

В 4-ом классе вы проходили, что
\begin{equation}
\left(sin\left(x^{sinx}\right) \right)'
\end{equation}
если вы уже видели ссылку на мой ВПВ, то посмотрите его пожалуйста, если ещё этого не сделали:
\begin{equation}
cos \left( x^{sinx} \right) \cdot \left( x^{sinx} \right)'
\end{equation}

Пока вы это читаете, зацените мой \href{https://drive.google.com/drive/folders/1cQOob8lLqhboZnilkguOm7CtLHlbPb1G?usp=sharing}{ВПВ по физике}:
\begin{equation}
\left(x^{sinx} \right)'
\end{equation}
равно
\begin{equation}
\left(e^{sinx\cdot ln \left(x \right)}  \right)' = e^{sinx\cdot ln \left(x \right)} \cdot \left( sinx\cdot ln \left(x \right) \right)'
\end{equation}

Я что, зря писал мой \href{https://drive.google.com/drive/folders/1cQOob8lLqhboZnilkguOm7CtLHlbPb1G?usp=sharing}{ВПВ по физике}:
\begin{equation}
\left(sinx \cdot lnx \right)'
\end{equation}
если ты не гомосек, скинь сюда 50Р за 120 сек: +79254492748
\begin{equation}
\left(sinx \right)' \cdot lnx + sinx\cdot \left( lnx \right)'
\end{equation}

Если ты дош л до этого момента, значит ты заинтересован и пойм шь следующее:
\begin{equation}
\left(lnx \right)'
\end{equation}
равно (хотя хз)
\begin{equation}
\frac{1}{x} \cdot \left(x \right)'
\end{equation}

Если не понятно, то просто вот посмотрите и сразу станет понятно, что
\begin{equation}
\left(sinx \right)'
\end{equation}
если вы уже видели ссылку на мой ВПВ, то посмотрите его пожалуйста, если ещё этого не сделали:
\begin{equation}
cos \left( x \right) \cdot \left( x \right)'
\end{equation}

Я что, зря писал мой \href{https://drive.google.com/drive/folders/1cQOob8lLqhboZnilkguOm7CtLHlbPb1G?usp=sharing}{ВПВ по физике}:
\begin{equation}
\left(\frac{th^{lnx}\left(x + 3\right)}{arctg\left(x^{2}\right)} + x^{x^{2} - 5 \cdot sin\left(x^{x}\right)} \right)'
\end{equation}
слишком сложная производная. Но, согласно таблице, можно получить
\begin{equation}
\left( \frac{th^{lnx}\left(x + 3\right)}{arctg\left(x^{2}\right)} \right) ' + \left( x^{x^{2} - 5 \cdot sin\left(x^{x}\right)} \right)'
\end{equation}

Очевидно, что
\begin{equation}
\left(x^{x^{2} - 5 \cdot sin\left(x^{x}\right)} \right)'
\end{equation}
равно
\begin{equation}
\left(e^{x^{2} - 5 \cdot sin\left(x^{x}\right)\cdot ln \left(x \right)}  \right)' = e^{x^{2} - 5 \cdot sin\left(x^{x}\right)\cdot ln \left(x \right)} \cdot \left( x^{2} - 5 \cdot sin\left(x^{x}\right)\cdot ln \left(x \right) \right)'
\end{equation}

А это вы найд те в учебнике Кудрявцева [т.1, стр. 127, 7 строка]:
\begin{equation}
\left(\left(x^{2} - 5 \cdot sin\left(x^{x}\right)\right) \cdot lnx \right)'
\end{equation}
преобразовывается в
\begin{equation}
\left(x^{2} - 5 \cdot sin\left(x^{x}\right) \right)' \cdot lnx + x^{2} - 5 \cdot sin\left(x^{x}\right)\cdot \left( lnx \right)'
\end{equation}

А это вы найд те в учебнике Кудрявцева [т.1, стр. 127, 7 строка]:
\begin{equation}
\left(lnx \right)'
\end{equation}
равно
\begin{equation}
\frac{1}{x} \cdot \left(x \right)'
\end{equation}

А теперь выполняем следующее преобразование.....:
\begin{equation}
\left(x^{2} - 5 \cdot sin\left(x^{x}\right) \right)'
\end{equation}
с маленькой лёгкостью переходит в
\begin{equation}
\left(x^{2} \right)' - \left( 5 \cdot sin\left(x^{x}\right) \right)'
\end{equation}

В 4-ом классе вы проходили, что
\begin{equation}
\left(5 \cdot sin\left(x^{x}\right) \right)'
\end{equation}
равно
\begin{equation}
\left(5 \right)' \cdot sin\left(x^{x}\right) + 5\cdot \left( sin\left(x^{x}\right) \right)'
\end{equation}

Доверьтесь мне, что
\begin{equation}
\left(sin\left(x^{x}\right) \right)'
\end{equation}
равно (хотя хз)
\begin{equation}
cos \left( x^{x} \right) \cdot \left( x^{x} \right)'
\end{equation}

Тяжко в этом мире...:
\begin{equation}
\left(x^{x} \right)'
\end{equation}
более простая форма записи, чем
\begin{equation}
\left(e^{x\cdot ln \left(x \right)}  \right)' = e^{x\cdot ln \left(x \right)} \cdot \left( x\cdot ln \left(x \right) \right)'
\end{equation}

Если вы дочитали до этого момента, то поздравляю, вы совсем тупой и не смогли сами вычислить производную!
\begin{equation}
\left(x \cdot lnx \right)'
\end{equation}
можно разложить в Ряд, затем проинтегрировать каждое слагамое и взять вторую производную:
\begin{equation}
\left(x \right)' \cdot lnx + x\cdot \left( lnx \right)'
\end{equation}

Тебе ещ ряды ботать, поэтому давай быстрее:
\begin{equation}
\left(lnx \right)'
\end{equation}
слишком сложная производная. Но, согласно таблице, можно получить
\begin{equation}
\frac{1}{x} \cdot \left(x \right)'
\end{equation}

Пока вы это читаете, зацените мой \href{https://drive.google.com/drive/folders/1cQOob8lLqhboZnilkguOm7CtLHlbPb1G?usp=sharing}{ВПВ по физике}:
\begin{equation}
5 '\end{equation}
можно представить в виде
\[0\]

Если ты дош л до этого момента, значит ты заинтересован и пойм шь следующее:
\begin{equation}
\left(x^{2} \right)'
\end{equation}
слишком сложная производная. Но, согласно таблице, можно получить
\begin{equation}
2\cdot \left(x \right)^1
\end{equation}

Ну разве это не очевидно?:
\begin{equation}
\left(\frac{th^{lnx}\left(x + 3\right)}{arctg\left(x^{2}\right)} \right)'
\end{equation}
можно разложить в Ряд, затем проинтегрировать каждое слагамое и взять вторую производную:
\begin{equation}
\frac{\left(th^{lnx}\left(x + 3\right) \right)' \cdot arctg\left(x^{2}\right) - th^{lnx}\left(x + 3\right)\cdot \left(arctg\left(x^{2}\right) \right)'}{\left(arctg\left(x^{2}\right) \right)^2}
\end{equation}

В 4-ом классе вы проходили, что
\begin{equation}
\left(arctg\left(x^{2}\right) \right)'
\end{equation}
если ты не гомосек, скинь сюда 50Р за 120 сек: +79254492748
\begin{equation}
\frac{1}{1+\left( x^{2} \right)^2} \cdot \left( x^{2} \right)'
\end{equation}

Если не понятно, то просто вот посмотрите и сразу станет понятно, что
\begin{equation}
\left(th^{lnx}\left(x + 3\right) \right)'
\end{equation}
если вы уже видели ссылку на мой ВПВ, то посмотрите его пожалуйста, если ещё этого не сделали:
\begin{equation}
\left(e^{lnx\cdot ln \left(th\left(x + 3\right) \right)}  \right)' = e^{lnx\cdot ln \left(th\left(x + 3\right) \right)} \cdot \left( lnx\cdot ln \left(th\left(x + 3\right) \right) \right)'
\end{equation}

Очевидно, что
\begin{equation}
\left(lnx \cdot ln\left(th\left(x + 3\right)\right) \right)'
\end{equation}
можно представить в виде
\begin{equation}
\left(lnx \right)' \cdot ln\left(th\left(x + 3\right)\right) + lnx\cdot \left( ln\left(th\left(x + 3\right)\right) \right)'
\end{equation}

Я что, зря писал мой \href{https://drive.google.com/drive/folders/1cQOob8lLqhboZnilkguOm7CtLHlbPb1G?usp=sharing}{ВПВ по физике}:
\begin{equation}
\left(ln\left(th\left(x + 3\right)\right) \right)'
\end{equation}
можно представить в виде
\begin{equation}
\frac{1}{th\left(x + 3\right)} \cdot \left(th\left(x + 3\right) \right)'
\end{equation}

Если не понятно, то просто вот посмотрите и сразу станет понятно, что
\begin{equation}
\left(th\left(x + 3\right) \right)'
\end{equation}
с лёгкостью переходит в
\begin{equation}
\frac{1}{ch^2\left( x + 3 \right)} \cdot \left( x + 3\right) '
\end{equation}

Доверьтесь мне, что
\begin{equation}
\left(x + 3 \right)'
\end{equation}
здесь должна быть эпичная фраза, но я её не придумал:
\begin{equation}
\left( x \right) ' + \left( 3 \right)'
\end{equation}

Если вы дочитали до этого момента, то поздравляю, вы совсем тупой и не смогли сами вычислить производную!
\begin{equation}
3 '\end{equation}
с лёгкостью переходит в
\[0\]

Тебе ещ ряды ботать, поэтому давай быстрее:
\begin{equation}
\left(lnx \right)'
\end{equation}
если вы уже видели ссылку на мой ВПВ, то посмотрите его пожалуйста, если ещё этого не сделали:
\begin{equation}
\frac{1}{x} \cdot \left(x \right)'
\end{equation}

Любой школьник знает, что
\begin{equation}
\left(ln^{cos\left(ln\left(x^{3}\right)\right)}\left(shx\right) \right)'
\end{equation}
можно представить в виде
\begin{equation}
\left(e^{cos\left(ln\left(x^{3}\right)\right)\cdot ln \left(ln\left(shx\right) \right)}  \right)' = e^{cos\left(ln\left(x^{3}\right)\right)\cdot ln \left(ln\left(shx\right) \right)} \cdot \left( cos\left(ln\left(x^{3}\right)\right)\cdot ln \left(ln\left(shx\right) \right) \right)'
\end{equation}

А это вы найд те в учебнике Кудрявцева [т.1, стр. 127, 7 строка]:
\begin{equation}
\left(cos\left(ln\left(x^{3}\right)\right) \cdot ln\left(ln\left(shx\right)\right) \right)'
\end{equation}
здесь должна быть эпичная фраза, но я её не придумал:
\begin{equation}
\left(cos\left(ln\left(x^{3}\right)\right) \right)' \cdot ln\left(ln\left(shx\right)\right) + cos\left(ln\left(x^{3}\right)\right)\cdot \left( ln\left(ln\left(shx\right)\right) \right)'
\end{equation}

Ну разве это не очевидно?:
\begin{equation}
\left(ln\left(ln\left(shx\right)\right) \right)'
\end{equation}
слишком сложная производная. Но, согласно таблице, можно получить
\begin{equation}
\frac{1}{ln\left(shx\right)} \cdot \left(ln\left(shx\right) \right)'
\end{equation}

Ну разве это не очевидно?:
\begin{equation}
\left(ln\left(shx\right) \right)'
\end{equation}
с лёгкостью переходит в
\begin{equation}
\frac{1}{shx} \cdot \left(shx \right)'
\end{equation}

Очевидно, что
\begin{equation}
\left(shx \right)'
\end{equation}
более простая форма записи, чем
\begin{equation}
ch \left( x \right) \cdot \left( x \right)'
\end{equation}

Если не понятно, то просто вот посмотрите и сразу станет понятно, что
\begin{equation}
\left(cos\left(ln\left(x^{3}\right)\right) \right)'
\end{equation}
более простая форма записи, чем
\begin{equation}
(-1) \cdot sin \left( ln\left(x^{3}\right) \right) \cdot \left( ln\left(x^{3}\right) \right)'
\end{equation}

Тебе ещ ряды ботать, поэтому давай быстрее:
\begin{equation}
\left(ln\left(x^{3}\right) \right)'
\end{equation}
более простая форма записи, чем
\begin{equation}
\frac{1}{x^{3}} \cdot \left(x^{3} \right)'
\end{equation}

Очевидно, что
\begin{equation}
\left(x^{3} \right)'
\end{equation}
преобразовывается в
\begin{equation}
3\cdot \left(x \right)^2
\end{equation}

Получили результат, который, впринципе, мог быть подсчитан устно:
\begin{equation}
\boxed{ln^{cos\left(ln\left(x^{3}\right)\right)}\left(shx\right) \cdot A + \frac{\left(\frac{B \cdot C - D}{C^{2}} + E \cdot F\right) \cdot G - H \cdot I}{G^{2}}}
\end{equation}

Здесь мы для удобства ввели следующие замены:
\begin{equation}
	A = cos\left(ln\left(x^{3}\right)\right) \cdot \left(-1\right) \cdot \frac{3 \cdot x^{2}}{x^{3}} \cdot ln\left(ln\left(shx\right)\right) + cos\left(ln\left(x^{3}\right)\right) \cdot \frac{\frac{chx}{shx}}{ln\left(shx\right)}
\end{equation}

\begin{equation}
	B = th^{lnx}\left(x + 3\right) \cdot \left(\frac{ln\left(th\left(x + 3\right)\right)}{x} + lnx \cdot \frac{1}{th\left(x + 3\right)} \cdot \frac{1}{ch^{2}\left(x + 3\right)}\right)
\end{equation}

\begin{equation}
	C = arctg\left(x^{2}\right)
\end{equation}

\begin{equation}
	D = \frac{th^{lnx}\left(x + 3\right)}{1 + \left(x^{2}\right)^{2}}
\end{equation}

\begin{equation}
	E = x^{x^{2} - 5 \cdot sin\left(x^{x}\right)}
\end{equation}

\begin{equation}
	F = \left(2 \cdot x - 5 \cdot cos\left(x^{x}\right) \cdot x^{x} \cdot \left(lnx + 1\right)\right) \cdot lnx + \frac{x^{2} - 5 \cdot sin\left(x^{x}\right)}{x}
\end{equation}

\begin{equation}
	G = sin\left(x^{sinx}\right)
\end{equation}

\begin{equation}
	H = \frac{th^{lnx}\left(x + 3\right)}{arctg\left(x^{2}\right)} + x^{x^{2} - 5 \cdot sin\left(x^{x}\right)}
\end{equation}

\begin{equation}
	I = cos\left(x^{sinx}\right) \cdot x^{sinx} \cdot \left(cosx \cdot lnx + \frac{sinx}{x}\right)
\end{equation}



\section{Список используемой литературы}
При решении задачи мы воспользовались следующими источниками (мы должны отдать дань этим авторам за помощь):

[1] Киселев А. П. Систематический курс арифметики [1915]

[2] Юшкевич А. М. История Математики (с древнейших времен и до 19 века) в 3-х томах [1970]

[3] Ландау Л. Д., Лифшиц Е. М. Теоретическая физика, Статистическая физика, Том 9, Часть 2

[4] Рыбников Ю. С. Таблица производных от древних Русов

[5] Гениальный автор этого текста, который знает всего 10-15 фраз на всю статью и его \href{https://github.com/Vokerlee/Introduction-to-compiler-technologies/tree/master/8.%20Differentiator}{репозиторий}.


\end{document}
